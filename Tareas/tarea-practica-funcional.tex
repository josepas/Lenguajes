Integrantes del grupo:

\begin{itemize}
\itemsep1pt\parskip0pt\parsep0pt
\item
  11-10428
  \href{mailto:gustavogut1993@gmail.com}{gustavogut1993@gmail.com}
\item
  11-10743 \href{mailto:josepas979@gmail.com}{josepas979@gmail.com}
\end{itemize}

\section{Introducción}\label{introducciuxf3n}

Este documento soporta la tarea práctica de programación funcional para
CI3661 (el Laboratorio de Lenguajes de Programación 1) en Abril--Julio
de 2015.

Puede cargar la fuente de este archivo en GHCi directamente con el
comando:

\begin{verbatim}
ghci tarea-practica-funcional
\end{verbatim}

\subsection{Declaraciones
preliminares}\label{declaraciones-preliminares}

En esta sección puede agregar todas las directivas necesarias para
importar símbolos de módulos adicionales a \texttt{Prelude} (que se
importa implícitamente) y para dar opciones al compilador.

\begin{Shaded}
\begin{Highlighting}[]
\OtherTok{\{-# LANGUAGE DeriveGeneric #-\}}
\OtherTok{\{-# LANGUAGE FlexibleInstances #-\}}
\end{Highlighting}
\end{Shaded}

\%\{-\# LANGUAGE LambdaCase \#-\}

\begin{Shaded}
\begin{Highlighting}[]
\OtherTok{\{-# LANGUAGE StandaloneDeriving #-\}}
\OtherTok{\{-# LANGUAGE TypeSynonymInstances #-\}}

\KeywordTok{import }\DataTypeTok{Control.Applicative} \NormalTok{(pure)}
\KeywordTok{import }\DataTypeTok{Control.DeepSeq}     \NormalTok{(}\DataTypeTok{NFData}\NormalTok{, ($!!))}
\KeywordTok{import }\DataTypeTok{Control.Monad}       \NormalTok{(void)}
\KeywordTok{import }\DataTypeTok{Data.Map}            \NormalTok{(}\DataTypeTok{Map}\NormalTok{, empty, singleton, fromList, foldWithKey)}
\KeywordTok{import }\DataTypeTok{GHC.Generics}        \NormalTok{(}\DataTypeTok{Generic}\NormalTok{)}
\KeywordTok{import }\DataTypeTok{System.Environment}  \NormalTok{(getArgs, getProgName)}
\KeywordTok{import }\DataTypeTok{System.IO}           \NormalTok{(hPutStrLn, stderr)}
\end{Highlighting}
\end{Shaded}

\section{Expresiónes aritméticas}\label{expresiuxf3nes-aritmuxe9ticas}

Considere la siguiente declaración de un tipo algebráico para
representar expresiones aritméticas con números de punto flotante:

\begin{Shaded}
\begin{Highlighting}[]
\KeywordTok{data} \NormalTok{Expresión}
  \NormalTok{= Suma           Expresión Expresión}
  \NormalTok{| Resta          Expresión Expresión}
  \NormalTok{| Multiplicación Expresión Expresión}
  \NormalTok{| División       Expresión Expresión}
  \NormalTok{| Negativo       Expresión}
  \FunctionTok{|} \DataTypeTok{Literal}        \DataTypeTok{Integer}
  \KeywordTok{deriving} \NormalTok{(}\DataTypeTok{Eq}\NormalTok{, }\DataTypeTok{Read}\NormalTok{, }\DataTypeTok{Show}\NormalTok{)}
\end{Highlighting}
\end{Shaded}

\begin{center}\rule{3in}{0.4pt}\end{center}

\textbf{Ejercicio 1} (0.2 puntos cada una; 0.6 puntos en total):
Complete las siguientes definiciones para que \texttt{t1}, \texttt{t2} y
\texttt{t3} representen a las respectivas expresiones aritméticas:

\begin{longtable}[c]{@{}lcl@{}}
\toprule\addlinespace
\texttt{t1} & = & \texttt{42}
\\\addlinespace
\texttt{t2} & = & \texttt{27 + t1}
\\\addlinespace
\texttt{t3} & = & \texttt{(t2 * (t2 * 1)) + (- ((t1 + 0) / 3))}
\\\addlinespace
\bottomrule
\end{longtable}

\begin{Shaded}
\begin{Highlighting}[]
\NormalTok{t1, t2, t3 :: Expresión}
\NormalTok{t1 }\FunctionTok{=} \DataTypeTok{Literal} \DecValTok{42}
\NormalTok{t2 }\FunctionTok{=} \DataTypeTok{Suma} \NormalTok{(}\DataTypeTok{Literal} \DecValTok{27}\NormalTok{) t1}
\NormalTok{t3 = Suma (Multiplicación t2 (Multiplicación t2 (}\DataTypeTok{Literal} \DecValTok{1}\NormalTok{))) }
          \NormalTok{(Negativo (División (}\DataTypeTok{Suma} \NormalTok{t1 (}\DataTypeTok{Literal} \DecValTok{0}\NormalTok{))(}\DataTypeTok{Literal} \DecValTok{3}\NormalTok{)))}
\end{Highlighting}
\end{Shaded}

\begin{center}\rule{3in}{0.4pt}\end{center}

\subsection{Catamorfismos}\label{catamorfismos}

Los valores de cada tipo algebraico en \emph{Haskell} son datos que
representan la estructura abstracta de una operación, sin especificar
cuál es la operación particular que debe hacerse. Un valor de un tipo
algebraico puede analizarse para convertirlo en una operación
particular: por ejemplo, el valor \texttt{t2} representa la estructura
abstracta de la operación «la suma de 42 y 27», sin decir qué significa
«la suma».

Un valor de esta forma puede convertirse en una operación particular de
varias maneras. Si bien la interpretación más obvia de esa operación
abstracta corresponde con la suma aritmética de los números 42 y 27, que
produce un número, también podrían realizarse \emph{otras} operaciones
siguiendo la misma estructura. Por ejemplo, puede buscarse cuál es el
máximo operando de todas las operaciones, en cuyo caso la operación
especificada como «suma» sería tomar el máximo entre dos números, y en
el caso de \texttt{t2} produciría como resultado el número 42.

En el contexto de \emph{Haskell}, un \textbf{catamorfismo} es cualquier
transformación de un tipo algebraico a una operación en otro tipo que se
haga de esta manera: según la estructura del valor del tipo algebraico.

\begin{center}\rule{3in}{0.4pt}\end{center}

\textbf{Ejercicio 2} (0.5 puntos): Complete la siguiente definición que
calcule el resultado de evaluar una expresión aritmética.

\%Aqui tenia Double entonces resolvi con fromInteger

\begin{Shaded}
\begin{Highlighting}[]
\NormalTok{evaluar :: Expresión }\OtherTok{->} \DataTypeTok{Double}
\NormalTok{evaluar e}
  \FunctionTok{=} \KeywordTok{case} \NormalTok{e }\KeywordTok{of}
      \DataTypeTok{Suma}           \NormalTok{e1 e2 }\OtherTok{->} \NormalTok{evaluar e1 }\FunctionTok{+} \NormalTok{evaluar e2}
      \DataTypeTok{Resta}          \NormalTok{e1 e2 }\OtherTok{->} \NormalTok{evaluar e1 }\FunctionTok{-} \NormalTok{evaluar e2}
      \NormalTok{Multiplicación e1 e2 }\OtherTok{->} \NormalTok{evaluar e1 }\FunctionTok{*} \NormalTok{evaluar e2}
      \NormalTok{División       e1 e2 }\OtherTok{->} \NormalTok{evaluar e1 }\FunctionTok{/} \NormalTok{evaluar e2}
      \DataTypeTok{Negativo}       \NormalTok{e     }\OtherTok{->} \NormalTok{evaluar e }\FunctionTok{*} \NormalTok{(}\FunctionTok{-}\DecValTok{1}\NormalTok{)}
      \DataTypeTok{Literal}        \NormalTok{n     }\OtherTok{->} \NormalTok{fromInteger n}
\end{Highlighting}
\end{Shaded}

En particular,

\begin{verbatim}
evaluar t1 == 42.0
evaluar t2 == 69.0
evaluar t3 == 4747.0
\end{verbatim}

\begin{center}\rule{3in}{0.4pt}\end{center}

\textbf{Ejercicio 3} (0.25 puntos): Complete la siguiente definición que
calcule la cantidad de operaciones aritméticas especificadas en una
expresión aritmética. Se considera que la expresión correspondiente a un
simple literal especifica cero operaciones.

\begin{Shaded}
\begin{Highlighting}[]
\NormalTok{operaciones :: Expresión }\OtherTok{->} \DataTypeTok{Integer}
\NormalTok{operaciones e }
  \FunctionTok{=} \KeywordTok{case} \NormalTok{e }\KeywordTok{of}
      \DataTypeTok{Literal}        \NormalTok{n     }\OtherTok{->} \DecValTok{0}
      \DataTypeTok{Negativo}       \NormalTok{e     }\OtherTok{->} \DecValTok{1} \FunctionTok{+} \NormalTok{operaciones e}
      \DataTypeTok{Suma}           \NormalTok{e1 e2 }\OtherTok{->} \DecValTok{1} \FunctionTok{+} \NormalTok{operaciones e1 }\FunctionTok{+} \NormalTok{operaciones e2}
      \DataTypeTok{Resta}          \NormalTok{e1 e2 }\OtherTok{->} \DecValTok{1} \FunctionTok{+} \NormalTok{operaciones e1 }\FunctionTok{+} \NormalTok{operaciones e2}
      \NormalTok{Multiplicación e1 e2 }\OtherTok{->} \DecValTok{1} \FunctionTok{+} \NormalTok{operaciones e1 }\FunctionTok{+} \NormalTok{operaciones e2}
      \NormalTok{División       e1 e2 }\OtherTok{->} \DecValTok{1} \FunctionTok{+} \NormalTok{operaciones e1 }\FunctionTok{+} \NormalTok{operaciones e2}
\end{Highlighting}
\end{Shaded}

En particular,

\begin{verbatim}
operaciones t1 == 0
operaciones t2 == 1
operaciones t3 == 8
\end{verbatim}

\begin{center}\rule{3in}{0.4pt}\end{center}

\textbf{Ejercicio 4} (0.25 puntos): Complete la siguiente definición que
calcule la suma de todos los literales presentes en una expresión
aritmética.

\begin{Shaded}
\begin{Highlighting}[]
\NormalTok{sumaLiterales :: Expresión }\OtherTok{->} \DataTypeTok{Integer}
\NormalTok{sumaLiterales e}
  \FunctionTok{=} \KeywordTok{case} \NormalTok{e }\KeywordTok{of}
      \DataTypeTok{Literal}        \NormalTok{n     }\OtherTok{->} \NormalTok{n}
      \DataTypeTok{Negativo}       \NormalTok{e     }\OtherTok{->} \NormalTok{sumaLiterales e}
      \DataTypeTok{Suma}           \NormalTok{e1 e2 }\OtherTok{->} \NormalTok{sumaLiterales e1 }\FunctionTok{+} \NormalTok{sumaLiterales e2}
      \DataTypeTok{Resta}          \NormalTok{e1 e2 }\OtherTok{->} \NormalTok{sumaLiterales e1 }\FunctionTok{+} \NormalTok{sumaLiterales e2}
      \NormalTok{Multiplicación e1 e2 }\OtherTok{->} \NormalTok{sumaLiterales e1 }\FunctionTok{+} \NormalTok{sumaLiterales e2}
      \NormalTok{División       e1 e2 }\OtherTok{->} \NormalTok{sumaLiterales e1 }\FunctionTok{+} \NormalTok{sumaLiterales e2}
\end{Highlighting}
\end{Shaded}

En particular,

\begin{verbatim}
sumaLiterales t1 == 42
sumaLiterales t2 == 69
sumaLiterales t3 == 184
\end{verbatim}

\begin{center}\rule{3in}{0.4pt}\end{center}

\textbf{Ejercicio 5} (0.25 puntos): Complete la siguiente definición que
calcule la lista de todos los literales presentes en una expresión
aritmética.

\begin{Shaded}
\begin{Highlighting}[]
\NormalTok{literales :: Expresión }\OtherTok{->} \NormalTok{[}\DataTypeTok{Integer}\NormalTok{]}
\NormalTok{literales e}
  \FunctionTok{=} \KeywordTok{case} \NormalTok{e }\KeywordTok{of}
      \DataTypeTok{Literal}        \NormalTok{n     }\OtherTok{->} \NormalTok{n}\FunctionTok{:}\NormalTok{[]}
      \DataTypeTok{Negativo}       \NormalTok{e     }\OtherTok{->} \NormalTok{literales e}
      \DataTypeTok{Suma}           \NormalTok{e1 e2 }\OtherTok{->} \NormalTok{literales e1 }\FunctionTok{++} \NormalTok{literales e2}
      \DataTypeTok{Resta}          \NormalTok{e1 e2 }\OtherTok{->} \NormalTok{literales e1 }\FunctionTok{++} \NormalTok{literales e2}
      \NormalTok{Multiplicación e1 e2 }\OtherTok{->} \NormalTok{literales e1 }\FunctionTok{++} \NormalTok{literales e2}
      \NormalTok{División       e1 e2 }\OtherTok{->} \NormalTok{literales e1 }\FunctionTok{++} \NormalTok{literales e2}
\end{Highlighting}
\end{Shaded}

En particular,

\begin{verbatim}
literales t1 == [42]
literales t2 == [27, 42]
literales t3 == [27, 42, 27, 42, 1, 42, 0, 3]
\end{verbatim}

\begin{center}\rule{3in}{0.4pt}\end{center}

\textbf{Ejercicio 6} (0.25 puntos): Complete la siguiente definición que
calcule la altura de una expresión aritmética. Se considera que un
literal es una expresión aritmética de altura cero, y que todas las
demás operaciones agregan uno a la altura.

\begin{Shaded}
\begin{Highlighting}[]
\NormalTok{altura :: Expresión }\OtherTok{->} \DataTypeTok{Integer}
\NormalTok{altura e}
  \FunctionTok{=} \KeywordTok{case} \NormalTok{e }\KeywordTok{of}
      \DataTypeTok{Literal}        \NormalTok{n     }\OtherTok{->} \DecValTok{0}
      \DataTypeTok{Negativo}       \NormalTok{e     }\OtherTok{->} \DecValTok{1} \FunctionTok{+} \NormalTok{altura e}
      \DataTypeTok{Suma}           \NormalTok{e1 e2 }\OtherTok{->} \DecValTok{1} \FunctionTok{+} \NormalTok{max (altura e1) (altura e2)}
      \DataTypeTok{Resta}          \NormalTok{e1 e2 }\OtherTok{->} \DecValTok{1} \FunctionTok{+} \NormalTok{max (altura e1) (altura e2)}
      \NormalTok{Multiplicación e1 e2 }\OtherTok{->} \DecValTok{1} \FunctionTok{+} \NormalTok{max (altura e1) (altura e2)}
      \NormalTok{División       e1 e2 }\OtherTok{->} \DecValTok{1} \FunctionTok{+} \NormalTok{max (altura e1) (altura e2)}
\end{Highlighting}
\end{Shaded}

En particular,

\begin{verbatim}
altura t1 == 0
altura t2 == 1
altura t3 == 4
\end{verbatim}

\begin{center}\rule{3in}{0.4pt}\end{center}

\subsection{Catamorfismo generalizado}\label{catamorfismo-generalizado}

Los catamorfismos de las preguntas anteriores deben seguir un patrón
común: cada constructor del tipo \texttt{Expresión} se hace corresponder
con una operación en el tipo del resultado del catamorfismo. Esas
operaciones se realizan con los valores almacenados en cada constructor,
y en el caso de las ocurrencias anidadas de valores del tipo
\texttt{Expresión}, éstas se transforman a valores del tipo del
resultado con una invocación recursiva al catamorfismo.

En efecto, todo catamorfismo se construye de la misma forma para un tipo
algebraico dado --- solo es necesario especificar de qué manera se
combinan a un valor del tipo resultante los datos obtenidos de cada
constructor.

\begin{center}\rule{3in}{0.4pt}\end{center}

\textbf{Ejercicio 7} (0.5 puntos): Complete la siguiente definición para
el catamorfismo generalizado del tipo \texttt{Expresión}.

\begin{Shaded}
\begin{Highlighting}[]
\NormalTok{cataExpresión}
\OtherTok{  ::} \NormalTok{(a }\OtherTok{->} \NormalTok{a }\OtherTok{->} \NormalTok{a)}
  \OtherTok{->} \NormalTok{(a }\OtherTok{->} \NormalTok{a }\OtherTok{->} \NormalTok{a)}
  \OtherTok{->} \NormalTok{(a }\OtherTok{->} \NormalTok{a }\OtherTok{->} \NormalTok{a)}
  \OtherTok{->} \NormalTok{(a }\OtherTok{->} \NormalTok{a }\OtherTok{->} \NormalTok{a)}
  \OtherTok{->} \NormalTok{(a }\OtherTok{->} \NormalTok{a)}
  \OtherTok{->} \NormalTok{(}\DataTypeTok{Integer} \OtherTok{->} \NormalTok{a)}
  \NormalTok{-> Expresión }\OtherTok{->} \NormalTok{a}

\NormalTok{cataExpresión}
  \NormalTok{suma}
  \NormalTok{resta}
  \NormalTok{multiplicacion}
  \NormalTok{division}
  \NormalTok{negativo}
  \NormalTok{literal}
  \NormalTok{e}
  \FunctionTok{=} \KeywordTok{case} \NormalTok{e }\KeywordTok{of}
      \DataTypeTok{Literal}        \NormalTok{n     }\OtherTok{->} \NormalTok{literal n}
      \DataTypeTok{Negativo}       \NormalTok{e     }\OtherTok{->} \NormalTok{negativo (currificada e)}
      \DataTypeTok{Suma}           \NormalTok{e1 e2 }\OtherTok{->} \NormalTok{suma (currificada e1) (currificada e2)}
      \DataTypeTok{Resta}          \NormalTok{e1 e2 }\OtherTok{->} \NormalTok{resta (currificada e1) (currificada e2)}
      \NormalTok{Multiplicación e1 e2 }\OtherTok{->} \NormalTok{multiplicacion (currificada e1) (currificada e2)}
      \NormalTok{División       e1 e2 }\OtherTok{->} \NormalTok{division (currificada e1) (currificada e2)}
   \KeywordTok{where} \NormalTok{currificada = cataExpresión suma resta multiplicacion division negativo literal}
\end{Highlighting}
\end{Shaded}

\begin{center}\rule{3in}{0.4pt}\end{center}

\textbf{Ejercicio 8} (0.2 puntos cada una; 1 punto en total): Complete
las siguientes definiciones para los catamorfismos que definió en las
preguntas anteriores, esta vez en términos de \texttt{cataExpresión}.

\begin{Shaded}
\begin{Highlighting}[]
\NormalTok{evaluar}\CharTok{' :: Expresión -> Double}
\NormalTok{evaluar}\CharTok{' = cataExpresión (+) (-) (*) (/) (negate) (fromInteger)   }

\NormalTok{operaciones}\CharTok{' :: Expresión -> Integer}
\NormalTok{operaciones}\CharTok{' = cataExpresión (aumentar) (aumentar) (aumentar) (aumentar) (+1) (*0)}
                  \KeywordTok{where} \NormalTok{aumentar x y }\FunctionTok{=} \DecValTok{1} \FunctionTok{+} \NormalTok{x }\FunctionTok{+} \NormalTok{y}
      
\NormalTok{sumaLiterales}\CharTok{' :: Expresión -> Integer}
\NormalTok{sumaLiterales}\CharTok{' = cataExpresión (+) (+) (+) (+) (id) (id)}

\NormalTok{literales}\CharTok{' :: Expresión -> [Integer]}
\NormalTok{literales}\CharTok{' = cataExpresión (++) (++) (++) (++) (id) (:[])}

\NormalTok{altura}\CharTok{' :: Expresión -> Integer}
\NormalTok{altura}\CharTok{' = cataExpresión (aumentar) (aumentar) (aumentar) (aumentar) (1+) (*0)}
              \KeywordTok{where} \NormalTok{aumentar x y }\FunctionTok{=} \DecValTok{1} \FunctionTok{+} \NormalTok{max x y}
\end{Highlighting}
\end{Shaded}

\begin{center}\rule{3in}{0.4pt}\end{center}

\section{Lenguajes de marcado}\label{lenguajes-de-marcado}

Los lenguajes de marcado descendientes de SGML se utilizan para
especificar documentos jerárquicos, donde el texto del documento se
incluye en \emph{elementos} que pueden anidarse y se clasifican según el
\emph{nombre de etiqueta} de cada uno. Además, los elementos pueden
asociarse con un diccionario de \emph{atributos} textuales identificados
por un nombre de atributo.

Puede representarse una versión simplificada de esta idea con los
siguientes tipos de datos algebraicos de \emph{Haskell}:

\begin{Shaded}
\begin{Highlighting}[]
\KeywordTok{type} \DataTypeTok{Atributos}
  \FunctionTok{=} \DataTypeTok{Map} \DataTypeTok{String} \DataTypeTok{String}

\KeywordTok{newtype} \DataTypeTok{Documento}
  \FunctionTok{=} \DataTypeTok{Documento} \DataTypeTok{Elemento}
  \KeywordTok{deriving} \DataTypeTok{Show}

\KeywordTok{data} \DataTypeTok{Elemento}
  \FunctionTok{=} \DataTypeTok{Elemento} \DataTypeTok{String} \DataTypeTok{Atributos} \NormalTok{[}\DataTypeTok{Elemento}\NormalTok{]}
  \FunctionTok{|} \DataTypeTok{Texto} \DataTypeTok{String}
  \KeywordTok{deriving} \DataTypeTok{Show}
\end{Highlighting}
\end{Shaded}

Los documentos completos están formados por un elemento como raíz. Los
elementos pueden tener un nombre de etiqueta, un diccionario de
atributos (usando el tipo \texttt{Map} definido en el módulo
\texttt{Data.Map}), y una lista de elementos anidados --- salvo en el
caso de los elementos que sencillamente contienen texto.

\begin{center}\rule{3in}{0.4pt}\end{center}

\subsection{Combinadores}\label{combinadores}

\textbf{Ejercicio 9} (0.15 puntos cada una; 0.6 puntos en total):
Complete las siguientes definiciones para combinadores que produzcan
representaciones de los elementos de XHTML \texttt{html}, \texttt{head},
\texttt{body} y \texttt{div} a partir de una lista de elementos anidados
dentro de ellos. Los elementos resultantes de aplicar estos combinadores
deben tener diccionarios de atributos vacíos, salvo en el caso del
elemento \texttt{html}, que debe incluir exactamente un atributo que
debe llamarse \texttt{xmlns} y asociarse al valor
\texttt{http://www.w3.org/1999/xhtml}.\footnote{Se utiliza el sufijo
  \texttt{E} para evitar conflictos con los nombres \texttt{head} y
  \texttt{div} importados implícitamente desde el módulo
  \texttt{Prelude} de \emph{Haskell}.}

\begin{Shaded}
\begin{Highlighting}[]
\NormalTok{htmlE, headE, bodyE,}\OtherTok{ divE ::} \NormalTok{[}\DataTypeTok{Elemento}\NormalTok{] }\OtherTok{->} \DataTypeTok{Elemento}
\NormalTok{htmlE  }\FunctionTok{=} \DataTypeTok{Elemento} \StringTok{"html"} \NormalTok{(singleton }\StringTok{"xmlns"} \StringTok{"http://www.w3.org/1999/xhtml"}\NormalTok{) }
\NormalTok{headE  }\FunctionTok{=} \DataTypeTok{Elemento} \StringTok{"head"} \NormalTok{empty }
\NormalTok{bodyE  }\FunctionTok{=} \DataTypeTok{Elemento} \StringTok{"body"} \NormalTok{empty}
\NormalTok{divE   }\FunctionTok{=} \DataTypeTok{Elemento} \StringTok{"div"}  \NormalTok{empty}
\end{Highlighting}
\end{Shaded}

\begin{center}\rule{3in}{0.4pt}\end{center}

\textbf{Ejercicio 10} (0.15 puntos cada una; 0.6 puntos en total):
Complete las siguientes definiciones para combinadores que produzcan
representaciones de los elementos de XHTML \texttt{title},
\texttt{style}, \texttt{h1} y \texttt{p} a partir de un \texttt{String}
con el texto que debe incluirse dentro de ellos. Los elementos
resultantes de aplicar estos combinadores deben tener diccionarios de
atributos vacíos, salvo el elemento \texttt{style} que debe tener el
atributo \texttt{type} asociado al texto \texttt{text/css}.

\begin{Shaded}
\begin{Highlighting}[]
\NormalTok{styleE, titleE,}\OtherTok{ h1E ::} \DataTypeTok{String} \OtherTok{->} \DataTypeTok{Elemento}
\NormalTok{styleE s }\FunctionTok{=} \DataTypeTok{Elemento} \StringTok{"style"} \NormalTok{(singleton }\StringTok{"type"} \StringTok{"text/css"}\NormalTok{) [}\DataTypeTok{Texto} \NormalTok{s]}
\NormalTok{titleE s }\FunctionTok{=} \DataTypeTok{Elemento} \StringTok{"title"} \NormalTok{empty [}\DataTypeTok{Texto} \NormalTok{s]}
\NormalTok{h1E    s }\FunctionTok{=} \DataTypeTok{Elemento} \StringTok{"h1"}    \NormalTok{empty [}\DataTypeTok{Texto} \NormalTok{s]}
\NormalTok{pE     s }\FunctionTok{=} \DataTypeTok{Elemento} \StringTok{"p"}     \NormalTok{empty [}\DataTypeTok{Texto} \NormalTok{s]}
\end{Highlighting}
\end{Shaded}

\begin{center}\rule{3in}{0.4pt}\end{center}

\textbf{Ejercicio 11} (0.2 puntos): Complete la siguiente definición
para un combinador que produzca una representación del elemento de XHTML
\texttt{p} a partir de un valor de cualquier tipo \texttt{a} que
pertenezca a la clase de tipos \texttt{Show}; el elemento \texttt{p}
resultante de aplicar este combinador debe contener únicamente un nodo
de texto cuyo \texttt{String} sea el resultante de aplicar la función
\texttt{show} al valor pasado como parámetro, y debe tener su
diccionario de atributos vacío.

\begin{Shaded}
\begin{Highlighting}[]
\OtherTok{showP ::} \DataTypeTok{Show} \NormalTok{a }\OtherTok{=>} \NormalTok{a }\OtherTok{->} \DataTypeTok{Elemento}
\NormalTok{showP }\FunctionTok{=} \NormalTok{pE }\FunctionTok{.} \NormalTok{show}
\end{Highlighting}
\end{Shaded}

\begin{center}\rule{3in}{0.4pt}\end{center}

\subsection{Generación de XHTML}\label{generaciuxf3n-de-xhtml}

Considere la siguiente clase de tipos:

\begin{Shaded}
\begin{Highlighting}[]
\KeywordTok{class} \DataTypeTok{RenderXHTML} \NormalTok{a }\KeywordTok{where}
\OtherTok{  render ::} \NormalTok{a }\OtherTok{->} \DataTypeTok{String}
\end{Highlighting}
\end{Shaded}

Se desea usar el método \texttt{render} para generar texto XHTML a
partir de un valor de cualquier tipo que sea instancia de esta clase de
tipos.\footnote{Note que esta clase es estructuralmente idéntica a la
  clase \texttt{Show} de \emph{Haskell}. Sin embargo, recuerde que el
  propósito de la clase \texttt{Show} es construir representaciones
  textuales de valores de \emph{Haskell} para asistir al programador a
  estudiar un tipo de datos y visualizar sus valores --- la clase
  \texttt{RenderXHTML}, en cambio, existe para generar código XHTML a
  partir de algunos tipos de datos que puedan convertirse de esa manera.}
Para convertir un \texttt{Documento} a su código XHTML, se declara que
\texttt{Documento} es una instancia de \texttt{RenderXHTML}:

\begin{Shaded}
\begin{Highlighting}[]
\KeywordTok{instance} \DataTypeTok{RenderXHTML} \DataTypeTok{Documento} \KeywordTok{where}
  \NormalTok{render (Documento raíz)}
    \NormalTok{= encabezado ++ render raíz}
    \KeywordTok{where}
      \NormalTok{encabezado}
        \FunctionTok{=} \NormalTok{unlines}
          \NormalTok{[ }\StringTok{"<?xml version='1.0' encoding='UTF-8'?>"}
          \NormalTok{, }\StringTok{"<!DOCTYPE html"}
          \NormalTok{, }\StringTok{"     PUBLIC '-//W3C//DTD XHTML 1.0 Strict//EN'"}
          \NormalTok{, }\StringTok{"     'http://www.w3.org/TR/xhtml1/DTD/xhtml1-strict.dtd'>"}
          \NormalTok{]}
\end{Highlighting}
\end{Shaded}

\begin{center}\rule{3in}{0.4pt}\end{center}

\textbf{Ejercicio 12} (1.25 puntos): Escriba una instancia de la clase
\texttt{RenderXHTML} para el tipo \texttt{Atributos}. Puede suponer que
las claves de los diccionarios de atributos únicamente contienen nombres
de atributos válidos para XHTML, y que los valores de atributos no
contienen entidades ilegales en XHTML ni comillas --- es decir, no es
necesario que se preocupe por escapar el texto obtenido del diccionario
de atributos.

El texto generado debe corresponder a la lista de atributos que ocurre
dentro de una etiqueta XHTML. Por ejemplo, para el diccionario de
atributos

\begin{longtable}[c]{@{}ll@{}}
\toprule\addlinespace
Nombre de atributo & Valor
\\\addlinespace
\midrule\endhead
foo & bar baz
\\\addlinespace
quux & meh
\\\addlinespace
wtf & wow://such.example.com/amaze/
\\\addlinespace
\bottomrule
\end{longtable}

debe generar el texto

\begin{verbatim}
foo='bar baz' quux='meh' wtf='wow://such.example.com/amaze/'
\end{verbatim}

El orden en que genere las especificaciones de atributos es irrelevante.

\begin{Shaded}
\begin{Highlighting}[]
\KeywordTok{instance} \DataTypeTok{RenderXHTML} \DataTypeTok{Atributos} \KeywordTok{where}
  \NormalTok{render }\FunctionTok{=} \NormalTok{foldWithKey f }\StringTok{""} 
           \KeywordTok{where} \NormalTok{f k v acc }\FunctionTok{=} \NormalTok{acc }\FunctionTok{++} \StringTok{" "} \FunctionTok{++} \NormalTok{k }\FunctionTok{++} \StringTok{"='"} \FunctionTok{++} \NormalTok{v }\FunctionTok{++} \StringTok{"'"}
\end{Highlighting}
\end{Shaded}

\begin{center}\rule{3in}{0.4pt}\end{center}

\textbf{Ejercicio 13} (1.25 puntos): Escriba una instancia de la clase
\texttt{RenderXHTML} para el tipo \texttt{Elemento}. Puede suponer que
los nombres de etiquetas de los elementos siempre son válidos para
XHTML, y que los nodos de texto no contienen entidades ilegales ni
caracteres reservados por XHTML --- es decir, no es necesario que se
preocupe por escapar el texto obtenido del elemento.

El texto generado debe corresponder a una etiqueta XHTML para el
elemento dado. Por ejemplo, para un elemento con el nombre de etiqueta
\texttt{welp}, y un hijo que sea un nodo textual con el texto
\texttt{trololololo}, debe generar el texto

\begin{verbatim}
<welp foo='bar baz' quux='meh' wtf='wow://such.example.com/amaze/'>trololololo</welp>
\end{verbatim}

o cualquier texto con el mismo significado en XHTML --- el espacio en
blanco, por ejemplo, es irrelevante.

\begin{Shaded}
\begin{Highlighting}[]
\KeywordTok{instance} \DataTypeTok{RenderXHTML} \DataTypeTok{Elemento} \KeywordTok{where}
  \NormalTok{render e }
    \FunctionTok{=} \KeywordTok{case} \NormalTok{e }\KeywordTok{of} 
          \DataTypeTok{Texto} \NormalTok{s }\OtherTok{->} \NormalTok{s}
          \DataTypeTok{Elemento} \NormalTok{s a xs }\OtherTok{->} \StringTok{"<"} \FunctionTok{++} \NormalTok{s }\FunctionTok{++} \NormalTok{render a }\FunctionTok{++} \StringTok{">"}  \FunctionTok{++} \NormalTok{(concat (map render xs))  }\FunctionTok{++} \StringTok{"</"} \FunctionTok{++} \NormalTok{s }\FunctionTok{++} \StringTok{">"} 
\end{Highlighting}
\end{Shaded}

\begin{center}\rule{3in}{0.4pt}\end{center}

\subsection{XHTML para expresiones}\label{xhtml-para-expresiones}

\textbf{Ejercicio 14} (1.25 puntos): Complete la siguiente definición
que convierta un valor dado del tipo \texttt{Expresión} en un valor del
tipo \texttt{Elemento} que represente a la estructura de la expresión
aritmética con un árbol de elementos XHTML.

Los literales numéricos deben representarse con elementos \texttt{p} que
contengan un nodo de texto con una representación textual del número.

Las operaciones aritméticas deben representarse con un elemento
\texttt{div} que contenga a los operandos transformados en elementos
hijos, y que indique la operación con un elemento \texttt{p} que
contenga un nodo de texto con el símbolo correspondiente a la operación
--- en el caso de operaciones binarias, como la suma, ubique el símbolo
entre los elementos hijos correspondientes a los operandos, y en el caso
del negativo, ubique el símbolo \texttt{-} antes que el elemento hijo.

Escriba su definición en términos de \texttt{cataExpresión} y utilice
los combinadores para elementos de XHTML que definió previamente.

\begin{Shaded}
\begin{Highlighting}[]
\NormalTok{expresionXHTML :: Expresión }\OtherTok{->} \DataTypeTok{Elemento}
\NormalTok{expresionXHTML e}
  \FunctionTok{=} \KeywordTok{case} \NormalTok{e }\KeywordTok{of}
      \DataTypeTok{Literal}        \NormalTok{n     }\OtherTok{->} \NormalTok{showP n}
      \DataTypeTok{Negativo}       \NormalTok{e     }\OtherTok{->} \NormalTok{divE [pE }\StringTok{"-"}\NormalTok{, expresionXHTML e]}
      \DataTypeTok{Suma}           \NormalTok{e1 e2 }\OtherTok{->} \NormalTok{divE [expresionXHTML e1, pE }\StringTok{"+"}\NormalTok{, expresionXHTML e2]}
      \DataTypeTok{Resta}          \NormalTok{e1 e2 }\OtherTok{->} \NormalTok{divE [expresionXHTML e1, pE }\StringTok{"-"}\NormalTok{, expresionXHTML e2]}
      \NormalTok{Multiplicación e1 e2 }\OtherTok{->} \NormalTok{divE [expresionXHTML e1, pE }\StringTok{"*"}\NormalTok{, expresionXHTML e2]}
      \NormalTok{División       e1 e2 }\OtherTok{->} \NormalTok{divE [expresionXHTML e1, pE }\StringTok{"/"}\NormalTok{, expresionXHTML e2]}
\end{Highlighting}
\end{Shaded}

Por ejemplo, el resultado de \texttt{expresiónXHTML t2} debería ser
igual al de

\begin{verbatim}
Elemento "div" empty
  [ Elemento "p" empty [Texto "27"]
  , Elemento "p" empty [Texto "+" ]
  , Elemento "p" empty [Texto "42"]
  ]
\end{verbatim}

donde \texttt{empty} viene del módulo \texttt{Data.Map}.

\begin{center}\rule{3in}{0.4pt}\end{center}

\textbf{Ejercicio 15} (1.25 puntos): Complete la siguiente definición
que convierta un valor dado del tipo \texttt{Expresión} en un valor del
tipo \texttt{Documento} que muestre información sobre la expresión
aritmética dada en un documento XHTML. Debe usar los combinadores
definidos en ejercicios previos para implantar esta función.

Se espera que el documento generado sea lo más parecido posible al
mostrado en la dirección
\url{https://ldc.usb.ve/~05-38235/cursos/CI3661/2015AJ/expresión.xhtml}\footnote{Ese
  ejemplo fue generado a partir de la expresión \texttt{t3}.}. En
particular, debe mostrar:

\begin{itemize}
\itemsep1pt\parskip0pt\parsep0pt
\item
  la representación textual de la expresión,
\item
  la estructura de la expresión convertida en elementos de XHTML usando
  la función \texttt{expresiónXHTML},
\item
  el resultado de la evaluación numérica de la expresión,
\item
  la altura de la expresión,
\item
  el número de operaciones de la expresión, y
\item
  la lista de todos los literales numéricos que ocurren en la expresión.
\end{itemize}

Antes de cada una de esas secciones, incluya un elemento \texttt{h1} con
el nombre de la sección.

\begin{Shaded}
\begin{Highlighting}[]
\NormalTok{expresiónDocumento :: Expresión }\OtherTok{->} \DataTypeTok{Documento}
\NormalTok{expresiónDocumento e = Documento (htmlE [headE [titleE }\StringTok{"Expresión"}\NormalTok{, styleE estilo], }
                                         \NormalTok{bodyE [h1E }\StringTok{"Expresión original"}\NormalTok{, showP e, }
                                                \NormalTok{h1E }\StringTok{"Estructura"}\NormalTok{, expresionXHTML e,}
                                                \NormalTok{h1E }\StringTok{"Valor"}\NormalTok{, showP (evaluar' e),}
                                                \NormalTok{h1E }\StringTok{"Altura"}\NormalTok{, showP (altura' e),}
                                                \NormalTok{h1E }\StringTok{"Número de operaciones"}\NormalTok{, showP (operaciones e),}
                                                \NormalTok{h1E }\StringTok{"Literales"}\NormalTok{, showP (literales' e)]}
                                  \NormalTok{])}
\end{Highlighting}
\end{Shaded}

La siguiente definición contiene el texto necesario a incluir en el
elemento \texttt{style} del documento a generar. El elemento
\texttt{style} con este contenido debe ser incluido en el elemento
\texttt{head} del documento generado.

\begin{Shaded}
\begin{Highlighting}[]
\OtherTok{estilo ::} \DataTypeTok{String}
\NormalTok{estilo}
  \FunctionTok{=} \NormalTok{unlines}
    \NormalTok{[ }\StringTok{"div, p \{"}
    \NormalTok{, }\StringTok{"  border: 1px solid black;"}
    \NormalTok{, }\StringTok{"  float: left;"}
    \NormalTok{, }\StringTok{"  margin: 1em;"}
    \NormalTok{, }\StringTok{"\}"}
    \NormalTok{, }\StringTok{"h1 \{"}
    \NormalTok{, }\StringTok{"  clear: both;"}
    \NormalTok{, }\StringTok{"\}"}
    \NormalTok{]}
\end{Highlighting}
\end{Shaded}

\begin{center}\rule{3in}{0.4pt}\end{center}

\section{Programa principal}\label{programa-principal}

Se define un programa principal para poder compilar y ejecutar este
archivo:

\begin{Shaded}
\begin{Highlighting}[]
\KeywordTok{deriving} \KeywordTok{instance} \NormalTok{Generic Expresión}
\KeywordTok{instance} \NormalTok{NFData Expresión}

\OtherTok{main ::} \DataTypeTok{IO} \NormalTok{()}
\NormalTok{main }\FunctionTok{=} \KeywordTok{do}
  \NormalTok{args }\OtherTok{<-} \NormalTok{getArgs}
  \KeywordTok{case} \NormalTok{args }\KeywordTok{of}
    \NormalTok{(nombreArchivo : expresión}\DataTypeTok{Texto} \FunctionTok{:} \NormalTok{_) }\OtherTok{->} \KeywordTok{do}
      \NormalTok{expresión <- pure $!! read expresión}\DataTypeTok{Texto}
      \NormalTok{writeFile nombreArchivo . render $ expresiónDocumento expresión}

    \NormalTok{_ }\OtherTok{->} \KeywordTok{do}
      \NormalTok{progName }\OtherTok{<-} \NormalTok{getProgName}
      \NormalTok{hPutStrLn stderr $ }\StringTok{"Uso: "} \NormalTok{++ progName ++ }\StringTok{" ARCHIVO.xhtml EXPRESIÓN"}
\end{Highlighting}
\end{Shaded}

Puede compilar la fuente de este archivo con el comando

\begin{verbatim}
ghc tarea-práctica-funcional.lhs -o expresión
\end{verbatim}

y probar su solución con, por ejemplo, el comando

\begin{verbatim}
./expresión expresión.xhtml 'Suma (Literal 42) (Literal 27)'
\end{verbatim}
